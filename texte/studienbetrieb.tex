\addchap{Studienbetrieb}

\minisec{Die Grundbegriffe des Studiums in kurzen Worten erklärt.}

Wer \glqq frisch\grqq\ aus der Schule kommt, kennt als Lehrform vor allem den Dialog.
Üblicherweise versucht der Lehrer in der Schule, auf die Denkweise und das Arbeitstempo der Schüler einzugehen, unterhält sich mehr mit ihnen, als dass er ihnen einen Vortrag hält.
Am Ende der Stunde hat zumindest ein großer Teil der Schüler den Stoff verstanden.
An der Uni gibt es diese Lehrmethode nicht - dafür aber einige andere, an die man sich auch gewöhnen kann.
Hier wird viel Wert auf Eigenständigkeit gelegt, ein \glqq an die Hand genommen werden\grqq\ wie in der Schule, gibt es nicht mehr.
Das ist nicht die einzige Neuerung, die im Studienalltag auf euch zukommt.

\minisec{Stundenplan}

An der Uni gibt ein so genanntes Lehrangebot, das kurz vor Beginn jedes Semesters veröffentlicht wird.
Ihr findet diese bereits nach Semestern sortierte Liste von Lehrveranstaltungen online auf der Seite der Fakultät \link{http://inf.tu-dresden.de/}.
Ab dem zweiten Semester besteht eure Aufgabe darin, aus dem Angebot einen Stundenplan zu basteln.
Für den Anfang bekommt ihr jedoch zum Eingewöhnen fertige Stundenpläne von uns, aus denen ihr dann einfach einen zur Einschreibung auswählen könnt.
Während Vorlesungen generell einen festen Termin haben, könnt ihr euch in die Übungen flexibel eintragen.
Schreibt euch bei jExam \link{http://jexam.inf.tu-dresden.de/} einfach für die Übungsstunden eurer Wahl ein.
Stellt ihr später jedoch fest, dass euer Übungsleiter die Qualitäten einer Schlaftablette aufweist oder euch die Übung zu voll ist, zögert nicht die Übung zu wechseln.


\minisec{Vorlesung}

In diesen Veranstaltungen erlebt ihr meistens Professoren live.
Die Zahl der Zuhörer ist in der Regel zehn Mal so groß wie die Anzahl der Schüler in einer Unterrichtsstunde. Dadurch kann natürlich nicht auf jeden Studenten eingegangen werden. Sollte euch jedoch etwas unklar sein, stellt ruhig Fragen. Oft versteht der Großteil der anderen auch nichts.
Außerdem ist es ratsam dem Stoff stets zu folgen, da die Stoffmenge oftmals gewaltig erscheint. Sich darüber zu beschweren ist sinnlos, da der Lehrplan für die Professoren mehr oder minder vorgeschrieben ist.
Gerade deshalb hat man nur 20 Wochenstunden, da für die Nachbereitung einer Vorlesung mindestens die gleiche Zeit veranschlagt werden sollte.
Allerdings beschweren solltet ihr euch über schlechte Tafelbilder, undeutliche und leise Aussprache, sowie mangelnde Vorbereitung der Vorlesung. 
Professoren sind meist nicht Professoren, weil sie gute Didaktiker sind, sondern weil sie gut forschen können.
Welche Vorlesung ihr in welchem Semester besuchen solltet, findet ihr im jeweiligen Studienablaufplan eures Studiengangs (Bachelor Informatik \link{https://www.inf.tu-dresden.de/content/study/regulations/download/ba-inf/2009/study.app.2.de.pdf}, Bachelor Medieninformatik \link{https://www.inf.tu-dresden.de/content/study/regulations/download/ba-minf/2009/study.app.2.de.pdf}, Diplom Informatik\link{http://www.inf.tu-dresden.de/content/study/regulations/download/inf/2010/study.de.pdf}) oder im Vorlesungsverzeichnis auf der Seite der Fakultät \link{http://www.inf.tu-dresden.de/index.php?node_id=2709&ln=de}.


\minisec{Übungen}

Übungen werden zu fast allen Vorlesungen angeboten und dienen dazu, Aufgaben zum aktuellen Vorlesungsstoff zu bearbeiten. Klausuren orientieren sich häufig an den Übungsaufgaben, deshalb solltet ihr die Übungen, die meist nicht vom Dozenten selbst gehalten werden, regelmäßig besuchen.
Das hat nämlich auch den Vorteil, dass man ja bekanntlich viele Dinge besser versteht, wenn man sie noch einmal aus einem anderen Mund erklärt bekommt.
Die jeweils aktuellen Übungsaufgaben findet ihr auf der Seite des jeweiligen Dozenten, oft unter den Stichworten Teaching oder Lehre.
Es wird erwartet dass ihr euch die Aufgaben bereits vor der Übung anschaut, um dann Lösungsansätze zu diskutieren und Fragen stellen zu können.


\minisec{Praktikum}

Das erste Praktikum erwartet euch bereits in den Semesterferien des ersten Semesters – plant euren Urlaub also lieber nicht zu schnell!
Dort werdet ihr alle im Einführungspraktikum – Robolab, sowie die Diplomer zusätzlich im Strategiespielpraktikum, euer Können unter Beweis stellen.
Ein ganzes Praktikumssemester ist nur für Diplomstudenten im 7. Semester Pflicht.
Natürlich ist es trotzdem empfehlenswert Praktika in den Semesterferien zu machen, das steigert nicht nur eure Jobchancen, sondern zeigt euch auch, ob eure Studienwahl tatsächlich die Richtige war.


\minisec{Prüfungen}

Direkt an die Vorlesungszeit schließt die Prüfungszeit an – das wohl Schwierigste im Leben eines Studenten.
Die genauen Prüfungstermine findet ihr für das Wintersemester meist etwa Anfang Januar auf der Homepage der Fakultät \link{http://inf.tu-dresden.de/} unter \glqq Aktuelles\grqq\ oder direkt beim Prüfungsamt \link{http://www.inf.tu-dresden.de/index.php?node_id=904&ln=de}.
Im Laufe des Semesters habt ihr die Gelegenheit euch dafür rechtzeitig einzuschreiben, dies erfolgt auch hier über jExam.
Dort habt ihr auch die Möglichkeit, euch bis zu drei Werk(!)tage vor der Prüfung wieder auszutragen. Dann könnt ihr die Prüfung auch in einem späteren Semester schreiben, was aber natürlich nicht zum Regelfall werden sollte.
Solltet ihr aufgrund eines Rücktritts innerhalb der Frist oder einer plötzlichen Erkrankung von der Prüfung ausscheiden, könnt ihr euch auf der Seite des Prüfungsamtes informieren, welche Nachweise (Atteste) ihr im Prüfungsamt innerhalb welcher Frist einreichen müsst \link{http://www.inf.tu-dresden.de/index.php?node_id=906\&ln=de}.
Prüfungen werden mit Noten bewertet, alles außer \textbf{5} ist bestanden und bestandene Prüfungen können nicht wiederholt werden.
Seid ihr durchgefallen, habt ihr die Möglichkeit die Prüfung innerhalb von zwei Semestern zu wiederholen. Erst wenn ihr auch die zweite Wiederholungsklausur versemmelt habt, werdet ihr exmatrikuliert.
Genauere Informationen findet ihr zu dieser Thematik stets in der Prüfungs- bzw. der Studienordnung, die ihr euch unbedingt mal angeschaut haben solltet.
Eine erste Mathe-Prüfung erwartet euch übrigens bereits im Dezember.


\minisec{Leistungsnachweise}
Um zu manchen Prüfungen überhaupt erst zugelassen zu werden, benötigt ihr sogenannte Leistungsnachweise bzw. Scheine (siehe Prüfungsordnung).
Ihr erhaltet einen Schein bei einem Praktikum oder bei Scheinklausuren.
Einschreibungen dazu erfolgen ebenfalls online über jExam.
Scheine unterscheiden sich von Prüfungen insofern, dass ihr unendlich oft versuchen könnt, einen Schein in einem Fach zu erhalten.
Aber Vorsicht: Scheine sind oft Voraussetzungen für Prüfungen und diese müssen bis zu einem bestimmten Zeitpunkt abgelegt sein.
In den meisten Fällen bestehen Vorleistungen allerdings aus der Abgabe einer bestimmten Anzahl an Übungsaufgaben.

\minisec{Sprachausbildung}\label{sec:sprachausbildung}

Es werden von der TUD Kurse für fast alle möglichen (und unmöglichen) Sprachen angeboten.
Zu diesem Zweck gibt es zwei Zentren für die Sprachausbildung: "Lehrzentrum Sprachen und Kulturen" (LSK) und "TUD Institute of Advanced Studies" (TUDIAS).
Das Sprachangebot der beiden Einrichtungen ähnelt sich sehr stark. Allerdings ist die Sprachausbildung am TUDIAS im Gegensatz zum LSK kostenpflichtig.
Ihr habt für diverse Sprachkurse ein Budget an Semesterwochenstunden (insgesamt 10 SWS), die ihr wie ihr wollt ausgeben könnt.
Für euer Studium zum Bachelor der (Medien-)Informatik sind Sprachkurse generell optional, aber auf jeden Fall empfehlenswert.
Für Diplomstudenten sind 4 SWS Englisch (also 2 Semester) Pflicht.
Die Einschreibung für einen Sprachkurs erfolgt online \link{https://sprachausbildung.tu-dresden.de} mit eurem ZIH-Login.
Sobald die Kurse freigeschaltet sind, solltet ihr euch jedoch stark beeilen, die beliebten Kurse sind meist innerhalb weniger Minuten voll.
Weitere Infos findet ihr unter \link{http://tu-dresden.de/die_tu_dresden/zentrale_einrichtungen/lsk/lskonline} und \link{http://www.tudias.de/de/Sprachschule.html}.
