\newcommand{\checkbox}[1]{$\square$\ \textbf{#1} \\}

\addchap{Erstsemester-Checkliste}

Für einen erfolgreichen Start in das Studium solltest du einige organisatorische Kleinigkeiten unbedingt in den ersten Wochen erledigen.
Diese haben wir dir in folgender Checkliste mit absteigender Priorität zusammengestellt.

\checkbox{Wohnung}
Solltest du noch keine Bleibe gefunden haben, ist Beeilung angesagt, die schönsten Wohnungen sind schnell weg.
Wenn du in den Genuss eines 10- bzw. 100-Mbit/s-Internetzugangs kommen möchtest, seien dir die Wohnheime \link{https://www.studentenwerk-dresden.de/wohnen/wohnheimkatalog} des Studentenwerks Dresden empfohlen. Alternativ bieten sich auch Portale wie \textit{WG-Gesucht} \link{https://wg-gesucht.de/} an.

\checkbox{Mail Account}
Siehe \textit{ZIH} in diesem Heft. Wichtig ist vor allem auch, das Erstpasswort zu ändern. Besuche hierfür am besten den Identity Manager des ZIH \link{https://idm-service.tu-dresden.de}.

\checkbox{E-Meal Karte}
Die Mensa Karte gibt es während der ESE oder in den Mensen selbst für jeweils 5\euro\ Pfand.
Zusätzlich dazu benötigst du deine E-Meal Bescheinigung, die du auf deinem Semesterbogen findest.

\checkbox{(optional) Sprachkurse}
Die TU bietet Sprachkurse für Englisch und viele weitere Sprachen an. Zur Einschreibung in den Informatik Master (nach dem Bachelor) müssen Englischkenntnisse nachgewiesen werden. Die Einschreibung für die Sprachkurse wird je nach Kurs im Laufe der ersten beiden Wochen deines Studiums freigeschaltet.
Erkundige dich auf den Seiten des LSK \link{https://lskonline.tu-dresden.de} frühzeitig, wann dies ist. Die meisten Kurse sind sehr schnell voll.

\checkbox{(optional) Sportkurse}
Wie für die Sprachkurse gilt auch hier, wer zuerst da ist...
Das Angebot kannst du beim Universitätssportzentrum (USZ) einsehen \link{https://tu-dresden.de/usz}.
Hast du dich für einen Kurs entschieden und bei freigeschalteter Einschreibung für diesen angemeldet, musst du nur noch die Anmeldebescheinigung drucken und den Kostenbeitrag innerhalb von drei Tagen auf das Konto des USZ überweisen.

\checkbox{Studienrelevante Dokumente}
Das Vorlesungsverzeichnis und die Prüfungs- und Studienordnung erhältst du auf den Seiten des Prüfungsamtes \link{https://tu-dresden.de/inf/pra}.
Gedruckte Ordnungen gibts beim FSR und in deiner ESE-Tüte.
Alle wichtigen Informationen zu den einzelnen Vorlesungen findest du auf den jeweiligen Seiten der Institute im Netz.
Die Professoren werden dir dazu jedoch auch noch alles in den ersten Vorlesungen mitteilen. Sonst hilft natürlich schon einmal ein Blick auf die Seite des FSR \link{https://www.ifsr.de}.

\checkbox{Wohnsitz anmelden}
Offiziell musst du innerhalb von zwei Wochen beim zuständigen Ortsamt \link{https://welcome.dresden.de/de/formulare.php} deine Wohnung anmelden.
Wer seinen Hauptwohnsitz nach Dresden verlegt bekommt von der Stadt eine \glqq Umzugsbeihilfe\grqq\ in Höhe von 150\euro.
Informationen dazu gibt's unter \link{https://www.dresden.de/de/rathaus/dienstleistungen/c_336.php} und \link{https://studentenwerk-dresden.de/wohnen/umzugsbeihilfe.html}.
Wenn die Wohnung als Nebenwohnsitz angemeldet wird, muss Zweitwohnsteuer gezahlt werden. Diese beträgt 10\% der Kaltmiete pro Monat. Weitere Informationen findest du beim StuRa \link{https://www.stura.tu-dresden.de/zweitwohnungssteuer} oder der Stadt Dresden \link{http://www.dresden.de/de/rathaus/dienstleistungen/c_zweitwohnungssteuer.php}.

\checkbox{BAföG Antrag}
Die Antragsformulare findest auf den Seiten des BMBF \link{https://www.bafög.de/de/alle-antragsformulare-432.php}. Für weitere Auskünfte steht dir das Servicebüro oder dein Sachbearbeiter im Studentenwerk zur Verfügung.
Schiebe den Antrag nicht allzu lang vor dir her, da dein Anspruch frühestens ab dem Antragsmonat gilt.
Informationen zu den Sprechzeiten beim Studentenwerk gibt es hier \link{https://www.studentenwerk-dresden.de/finanzierung/servicebuero.html}.

\checkbox{Bibliotheksausweis}
Bekommt man direkt am Schalter in der SLUB (Zellescher Weg 18) \link{https://www.slub-dresden.de/service/anmelden}.

\checkbox{Programmierkurse}
Besonders denjenigen ohne Programmiererfahrung werden die im Wintersemester angebotenen C und Java-Kurse ans Herz gelegt.
Diese finden in der Regel unter der Woche statt, manche auch am Wochenende.
Für Details wende dich an \link{programmierung@ifsr.de}, \link{fredo@ifsr.de} und behalte die News auf der Seite des FSR \link{https://www.ifsr.de} im Auge.

\checkbox{Fachschaftsratwahlen}
Wähle deine studentischen Vertreter im FSR Informatik.
Die Wahlen finden jedes Jahr im November statt.
Geh wählen!
Und noch besser: Lass dich wählen!

\checkbox{Prüfungseinschreibung}
Ab Anfang nächsten Jahres kann man sich in jExam für die Prüfungen anmelden. Der Termin wird auf der Seite des Prüfungsamtes bekannt gegeben. Melde dich rechtzeitig an, denn sonst kannst du nicht an der Prüfung teilnehmen.
Schreib dich in die Prüfungen der Fächer ein, die du besucht hast.
Beachte, dass die erste Prüfung in Mathe bereits Anfang Dezember statt findet.
Viel Erfolg!

\checkbox{Rückmeldung zum Sommersemester}
Ab Mitte Januar 2015 kannst du den Semesterbeitrag für das nächste Semester überweisen.
Den genauen Betrag und Termine findest du auf dem aktuellen Semesterbogen und hier \link{https://tu-dresden.de/imma/rueckmeldung}.
Kümmere dich rechtzeitig darum, sonst wirst du automatisch exmatrikuliert!

\vfill

\includegraphics[width=\linewidth]{img/ese2013/barschoe.jpg}

\vfill

\begin{figure}[h!]
\centering
\includegraphics[scale=.5]{img/xkcd/compiling.png}
\caption*{{\small \textit{'Are you stealing those LCDs?' 'Yeah, but I'm doing it while my code compiles.' (https://xkcd.com/303)}}}
\end{figure}
