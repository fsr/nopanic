% toggles back at the end of this file!
\changemenucolor{gray}{bg}{named}{ese_bg_color} %background of the menukeys
\changemenucolor{gray}{br}{named}{ese_fg_color} %border of the menukeys
\changemenucolor{gray}{txt}{named}{ese_fg_color} %text of the menukeys

\newcommand{\semester}[1]{\minisec{\Large\vspace{.7\baselineskip} #1. Semester\\[.6\baselineskip]}}
\newcommand{\modul}[1]{\vspace{.5\baselineskip}\textbf{\menu[;]{#1\,}}\\[.2\baselineskip]}

\addchap{Modulübersicht}

Hier findest du eine kurze Übersicht der Module, die du im Laufe deines Studiums besuchen wirst. Die Reihenfolge ist nicht bindend, es ist lediglich der empfohlene Weg durch das Studium.
Ein (M) kennzeichnet ein Modul nur für Medieninformatiker, ein (I) jeweils Module für Informatiker und ein (D) für Diplominformatiker.

\semester{1}

\modul{I; M; D; Einführung in die Mathematik für Informatiker}
Du kennst dich mit Matrizen aus?
Dann weißt du auch was mit den Begriffen Determinante, Diagonalisierbarkeit, Skalarprodukt und der Lösung eines homogenen linearen Gleichungssystems anzufangen - wenn nicht, dann lernst du es hier von der Pike auf.
Außerdem wird in der Diskreten Mathematik das Mal und Plus quasi neu definiert und du lernst ein wenig anders zu denken.

\modul{I; M; D; Algorithmen und Datenstrukturen}
Was kommt zuerst?
5 oder 3?
Solche Fragen werden dich in Algorithmen und Datenstrukturen beschäftigen, während du Quicksort, Heapsort und Konsorten kennenlernst.
% Der Wortwitz gefällt mir...
Außerdem wirst du dich als Gärtner versuchen, indem du AVL- und andere Bäume wachsen lasst.
Dabei wirst du Bekanntschaft mit der Programmiersprache C machen.

\modul{I; M; D; Einführungspraktikum}
Du hast schon immer gerne mit Lego gespielt?
Dann wird dir dieses Praktikum, welches in der vorlesungsfreien Zeit stattfindet, gefallen.
Du darfst dich im Team daran machen, einem selbst konstruierten Roboter in Python beizubringen, wie er sich in einem Labyrinth alleine zurechtfindet.
Dabei, und im anschließenden Wettbewerb, kommt der Spaß nicht zu knapp.
Für Diplomstudenten gibt es zusätzlich ein einwöchiges Einzelprojekt, bei dem man zeigen kann, was man in C oder wahlweise C++ drauf hat. Alternativ kann
das Strategiespielepraktikum auch auf das Sommersemester verschoben werden.
In den letzten Jahren war eine KI für bekannte Brettspiele gefragt.

\newpage

\modul{I; M; Einführung in die Medieninformatik}
Anfangs erfolgt eine Darstellung des menschlichen Wahrnehmungssystems, Aspekte der Wahrnehmungspsychologie und der Softwareergonomie.
Dann werden Eigenschaften der Information und Datenformate anhand der Medien Text, Bild, Audio und Video dargestellt.
Im Bereich Text und Bild werden die entsprechenden Dokumentenformate des Internets (HTML und SVG) besprochen.
Außerdem erwartet dich ein kurzer Exkurs in die App-Entwicklung.
Ein weiterer Teil der Lehrveranstaltung gibt einen Überblick zur Dokumentenverarbeitung mittels XML-Techniken.
In Übungen und in Form eines Projektes in einer kleinen Gruppe über das Semester hinweg hast du die Möglichkeit, das Erlente direkt in die Praxis umzusetzen.

\modul{D; Technische Grundlagen und Hardwarepraktikum}
Wer schon immer mal wissen wollte, was die Elektronen im häuslichen Rechner eigentlich so alles durchmachen müssen, bekommt genau das hier vermittelt.
Anfangs werden Transistor-, Dioden- und Operationsverstärkerschaltungen betrachtet.
Darauf aufbauend werden Verknüpfungsglieder und komplexe Schaltungen näher unter die Lupe genommen.
Im folgenden Semester kannst du all das hier gelernte dann praktisch anwenden.

\modul{D; Rechnerarchitektur}
Hier geht es um die Grundbausteine eines Computers:
Speicher, Bussysteme, Rechen- und Steuerwerk.
Binärcode lesen? Alles Quatsch! In diesem Modul lernst du, wie Maschinensprache 
wirklich aussieht und bekommst einen kleinen Crash-Kurs in Assembler.
Außerdem schaust du dir das Pipelining-Prinzip an und wirst mit den damit auftretenden Problemen konfrontiert.
Schließlich wird noch diskutiert, mit welchen Methoden man heutige Rechnerarchitekturen beschleunigen und parallele Architekturen nutzen kann.

\vfill

\begin{figure}[h!]
\centering
\includegraphics[scale=.5]{img/xkcd/exploits_of_a_mom.png}
\caption*{{\small \textit{Her daughter is named Help I'm trapped in a driver's license factory. (https://xkcd.com/327)}}}
\end{figure}

\newpage

\semester{2}

\modul{I; M; D; Mathematische Methoden für Informatiker}
Nachdem der Abistoff viel tiefer als vorher sitzt, geht es in den nächsten zwei Semestern in neue Bereiche der Mathematik.
Anfangs werden die verschiedenen Typen algebraischer Strukturen (das sind Mengen von beliebigen Symbolen und darauf erklärte Rechenoperationen) untersucht.
Es folgen Vektoren, Matrizen und mathematische Körper.
Dann kommt ein Sprung vom Diskreten zum Kontinuierlichen.
So langweilig wie in der Schule ist Analysis nämlich gar nicht, die gibt es auch in der Ausführung mit mehreren Veränderlichen.
Das Ganze gipfelt in der Einführung von Differentialgleichungen.
Gegen Schluss wendet man sich erneut den Polynomen zu.
Dabei werden zunächst effiziente Näherungsverfahren behandelt, später folgt noch ein kurzer Ausflug in die Stochastik.

\modul{I; M; D; Programmierung}
Dass Programmiersprachen nicht auf Bäumen wachsen, wusstest du wahrscheinlich schon, doch dass sie strengen mathematischen Regeln folgen, lernst du hier.
Am Beispiel eines Teils der Programmiersprache C wird zunächst die Syntax mit Hilfe von Grammatiken definiert.
Kurz darauf kommst du mit der funktionalen Programmiersprache Haskell in Kontakt und lernst so einen ganz neuen Programmieransatz kennen.
Durch viele hübsche, rekursiv verschachtelte Abbildungen wird dann die Semantik festgelegt, d.h. die Wirkung, die so ein Programm auf einer (abstrakten) Rechenmaschine hat.
Hier wird auch vermittelt, wie man die Korrektheit eines Programmstückes formal beweist.

\modul{I; M; D; Informations- und Kodierungstheorie}
Was Informationen eigentlich sind und was sie ausmacht, wird dich hier beschäftigen.
In dieser Lehrveranstaltung wirst du einen Einstieg in ein sehr interessantes und komplexes Fachgebiet erhalten.
Im Mittelpunkt steht am Anfang, wie man Informationen darstellen und speichern kann.
Etwas später wird erklärt, warum und wie die Informationen mittels Kodierung geschützt werden, damit sie bei dir sicher ankommen, wenn sie unterwegs Störungen und Manipulationen ausgesetzt sind.
Dabei wird dir dein bisher in der Mathematik erworbenes Wissen von Nutzen sein.

\modul{D; Rechnerarchitektur}
Fortsetzung aus dem 1. Semester.

\newpage

\modul{D; Technische Grundlagen und Hardwarepraktikum}
Fortsetzung aus dem 1. Semester. Schaltskizzen und physikalische Gesetze -- schön und gut. Aber wie passt das alles zusammen?
Darum soll es in diesem Modul gehen. In einer Reihe von Experimenten wirst du all das gelernte mal praktisch anwenden.
Von analogen über digitale bis hin zu einem eigenen kleinen Computer ist alles mit dabei!

\modul{I; M; D; Softwaretechnologie}
Software zu entwickeln ist eine Kunst und gute Software zu schreiben keine einfache Sache, das wirst du spätestens nach diesem Modul erkennen.
Um diese Kunstfertigkeit an den Tag legen zu können, bedarf es einiger Handwerkszeuge, welche du hier mit auf den Weg bekommst.
So werden dir moderne Konzepte am Beispiel von Java und Entwurfsverfahren zusammen mit professioneller Dokumentation nähergebracht.
Damit wird dann der Grundstein für das Projekt im dritten Semester gelegt, bei dem man sich Lorbeeren im Projektmanagement und als Entwickler verdienen kann.

\modul{I; M; Einführung in die Computergraphik}
Was steckt eigentlich hinter der Unreal Engine? Wie funktionieren Shader? 
Wieso sehen die Figuren in Computerspielen immer realistischer aus?
Das erfährst du in diesem Modul, zusammen mit dem Aufbau von Grafiksystemen, Farbräumen, Rastergrafiken und deren Anwendungen.
Bestehende Probleme wie Aliasing und Artefakte sind mit von der Partie, sowie ihre algorithmischen Lösungen.
Als Programmiersprache für die praxisnahen Übungsaufgaben wird C++ genutzt.

\semester{3}

\modul{I; M; D; Mathematische Methoden für Informatiker}
Fortsetzung aus dem 2. Semester.

\modul{I; M; D; Formale Systeme}
Wahr?
Und oder falsch?
Was falsch ist wird, wenn es falsch falsch ist, wahr?
Logisch!
Wenn morgen alle einen Schirm dabei haben, wird es regnen?
Neben der Aussagenlogik vermittelt dieses Modul die Grundlagen formaler Sprachen.
Es folgen Gedanken zur maschinellen Berechenbarkeit und zur Automatentheorie.
Turing lässt grüßen.

\newpage

\modul{I; M; D; Softwaretechnologie-Projekt}
Das Projekt nimmt den größten Teil des dritten Semesters ein.
Hier muss man sein Wissen aus der Lehrveranstaltung \glqq Softwaretechnologie\grqq\ in die Tat umsetzen.
In einem fünfköpfigen Team hast du die Aufgabe, eine Anwendung für einen realen Kunden oder den Lehrstuhl zu schreiben.
Dabei müsst ihr euch als Team um die Konzeption, Planung, Programmierung und Arbeitsteilung kümmern.
Dabei müsst ihr auch Rücksprache mit den Kunden halten und vielleicht noch mal alles ändern.
Abgeschlossen wird das Modul mit der Präsentation des fertigen Produkts vor dem Kunden und den Verantwortlichen des Moduls.
Am Ende hast du dann einen Eindruck, wie die Arbeit eines Informatikers aussehen kann.

\modul{I; M; Rechnerarchitektur}
Beschreibung siehe 1. Semester.

\modul{I; Technische Grundlagen und Hardwarepraktikum}
Beschreibung siehe 1. Semester.

\modul{M; Grundlagen der Gestaltung}
Die Vorlesung beginnt mit Begriffsdefinitionen sowie allgemeinen Gestaltungsprinzipien und erläutert diese.
Dabei beschränkt sich die Veranstaltung bewusst auf zweidimensionale Bereiche.
Formkategorien, Kontrastbildung und Farblehre bilden die Schwerpunkte.
Die begleitenden Übungen sollen einen Einblick in die Materie vermitteln und die Sensibilität der Studierenden durch handwerkliches Arbeiten wecken.

\modul{D; Grundlagen des Nebenfachs}
Je nachdem, was du dir als Nebenfach wählst, beschäftigst du dich hier mit Themen, die nur im entfernten Sinne mit Informatik zusammenhängen.
Über den Tellerrand schauen und andere Welten kennenlernen ist das Motto.

\modul{D; Betriebssysteme und Sicherheit}
Beschreibung siehe 5. Semester.

\semester{4}

\modul{I; M; Rechnerarchitektur}
Fortsetzung aus dem 3. Semester.

\modul{I; D; Theoretische Informatik und Logik}
Die Fortsetzung der Formalen Systeme.
Es folgen weitere Betrachtungen zur Korrektheit und Terminierung von Algorithmen und dem notwendigen Aufwand in Form von Zeit und Platzbedarf.
Ein Abstecher in die Prädikatenlogik und Logikprogrammierung rundet das Modul ab.

\modul{I; M; D; Datenbanken und Rechnernetze}
Wohin mit meinen 20 Terabyte Nutzerdaten? Und wie kommt das Youtube-Video eigentlich von den USA in meinen Browser?
Darum wird es in diesem Modul gehen, das aus zwei verschiedenen Lehrveranstaltungen besteht.
In \textit{Datenbanken} lernt man zuerst Methoden zur effizienten Datenspeicherung kennen.
Danach wird die Fähigkeit vermittelt, selbst komplexe relationale Datenbanken zu konzipieren und zu erstellen.
In \textit{Rechnernetze} fängt man mit dem Funktionsprinzip von Modem und Netzwerkkarte an und erhält einen kurzen Überblick über moderne Kommunikations- und Vermittlungsprotokolle.
Auch der Sektor Mobilkommunikation und die dabei auftretenden Schwierigkeiten werden kurz beleuchtet.

\modul{I; Technische Grundlagen und Hardwarepraktikum}
Fortsetzung aus dem 3. Semester.

\modul{M; Einführung in die Mediengestaltung}
Die Vorlesung vermittelt die Grundzüge des multimedialen Gestaltens unter Gesichtspunkten der Entwicklung der einzelnen Richtungen (Film, Internet) mit Bezug auf die gestalterischen Änderungen in den vergangenen Jahrhunderten (Buch).
Außerdem wird in die Metaphernbildung eingeführt und es werden Schwerpunkte in Richtung des Interfacedesign gesetzt.

\modul{M; Medien und Medienströme}
Hier wird Wissen zu Medien, deren Kompression und Bearbeitung vermittelt.
Die Anwendung verschiedener Werkzeuge zur Erzeugung von Medien und deren Charakteristika sind ebenfalls Gegenstand dieser Lehrveranstaltung.
So wirst du dich in Form von Übungsaufgaben mit den Grundlagen der Bild-, Audio- und Videobearbeitung auseinandersetzen.

\modul{M; Medienpsychologie und -didaktik}
Mediendidaktik ist die \glqq Kunst des Lehrens\grqq.
Hier werden die Fragen beantwortet:
Was ist Bildung?
Wie verläuft sie?
Wie lässt sie sich vervollkommnen?
Man erfährt etwas über die Entwicklung von Lehrmethoden.
Im parallel stattfindenden Praktikum wendest du das Gelernte dann beim Entwickeln eines Lernspiels an.

\modul{M; Komplexpraktikum}
Das große Highlight für Medieninformatiker im Bachelor.
In kleineren Gruppen soll ein mobiles Spiel, eine Internet-Seite oder etwas anderes Multimediales realisiert werden.
Abgesehen von der Aufgabenstellung sind der Fantasie quasi keine Grenzen gesetzt.
Es geht um harte Arbeit, Teamgeist und den gekonnten Umgang mit Schlafmangel, wenn die Deadline schließlich näher rückt.

\modul{D; Forschungslinie}
Hier erhältst du einen Überblick über aktuelle Forschungsthemen und bekommst vermittelt, wie man forschungsorientiert arbeitet.
Dieses Modul hilft, später die richtige Vertiefung zu wählen. Der Besuch dieser Veranstaltung ist auch für Bachelor- und Masterstudenten interessant, vertiefen müssen sie sich schließlich ebenfalls.

\modul{D; Grundlagen des Nebenfachs}
Fortsetzung aus dem 3. Semester.

\modul{D; Allg. Basisqualifikationen}
Englisch ist die einzig relevante Sprache in der Informatik.
Hier wird dir vermittelt, wie man sich fachlich auf Englisch ausdrückt. 
Zusätzlich zu den zwei verpflichtend zu besuchenden Sprachkursen gehört noch ein Proseminar dazu.
Dort wirst du lernen, \textit{wie} man eine wissenschaftliche Veröffentlichung anfertigt -- indem du selbst eine zu einem Thema deiner Wahl schreibst und präsentierst.

\begin{figure}[b!]
\centering
\includegraphics[scale=.4]{img/xkcd/vomiting_emoji.png}
\caption*{{\small \textit{My favorite might be U+1F609 U+1F93F WINKING FACE VOMITING. (https://xkcd.com/1813)}}}
\end{figure}

\newpage

\semester{5}

\modul{I; M; Betriebssysteme und Sicherheit}
Diese Lehrveranstaltung nimmt die dienstbaren Geister, die zwischen der Hardware und den bunten Anwendungen werkeln, unter die Lupe.
Warum kann man mit einem Rechner gleichzeitig einen Text schreiben, Code kompilieren, ein Bild bearbeiten und Musik hören?
Wie werden meine Daten in Rechnersystemen geschützt?
Wieso stehen die hier auf dieses Unix?

\modul{I; D; Systemorientierte Informatik/Hardware Software Codesign}
Dieses Fachgebiet ist die Schnittstelle zwischen Rechnern und der industriellen Praxis, die von der Steuerung von Heizventilen bis zu Kraftwerken reicht.
Zunächst wird abstrahiert, was allen praktisch vorkommenden Systemen gemein ist, und es werden Modelle wie \glqq System\grqq, \glqq Signal\grqq\ und \glqq Regelkreis\grqq\ erschaffen, mit denen sich dann rechnerisch umgehen lässt.
Hier wird man fit gemacht für die Analyse und Voraussage von Übertragungsverhalten und Reaktionen, die ein solches System bei einem bestimmten Input zeigen wird.
Daneben kommen auch Aspekte aus der Audio- und Videotechnik wie Digitalisierung und Filteralgorithmen nicht zu kurz.

\modul{I; D; Intelligente Systeme}
In dieser Lehrveranstaltung geht es um künstliche Intelligenz.
Hier erlernt man Problemlösung, Wissensrepräsentation, Planung, Wahrnehmung und Sprachverstehen mit Hilfe spezieller Algorithmen und Agenten.
Warum konnte IBM's Watson in \textit{Jeopardy!} gegen die besten menschlichen Spieler gewinnen? Woher weiß mein Spam-Filter, was Spam ist und was nicht?
Auch solche Fragen kommen hier nicht zu kurz, wenn es um die Lernalgorithmen hinter diesen Phänomenen geht.

\modul{M; Web- und Multimedia Engineering}
Wie kann man das Web mit heutiger Technik multimedial und interaktiv gestalten? Warum ist HTML5 so toll?
Wie nutze ich professionelle Entwicklungswerkzeuge und geeignete Sprachen, wie z.B. Java, um meine Vorstellung in das Ergebnis zu projizieren?
% professionelle Entwicklungswerkzeuge und geeignete Sprachen (im Bezug auf Webentwicklung) -> Java... Alles klar m(
Dieses Modul hilft geeignete Methoden zu erlernen und Erfahrung bei der Anwendung zu sammeln.

\modul{M; Komplexpraktikum}
Fortsetzung aus dem 4. Semester.

\newpage

\modul{I; M; D; Vertiefung}
Hier kannst du aus einem Angebotskatalog geeignete Veranstaltungen wählen, um deinen wissenschaftlichen Horizont zu erweitern.
Die Möglichkeiten umfassen Vorlesungen, Übungen, Praktika, Projektbearbeitungen, Exkursionen, Proseminare und mehr.

\modul{D; Vertiefung im Nebenfach}
Nachdem du dir die Grundlagen deines gewählten Nebenfachs angeeignet hast, wird es nun ernst und du steigst tiefer in die Materie ein.

\modul{D; Basismodul 1, 2 und 3}
Hier wählst du unter sieben verschiedenen Themenkomplexen drei aus und beschäftigst dich mit ihnen.
Zur Wahl stehen Angewandte Informatik, Künstliche Intelligenz, Software- und Web-Engineering, Systemarchitektur, Technische Informatik, Theoretische Informatik sowie Graphische Datenverarbeitung.
Innerhalb dieser Richtungen stehen verschiedene Vorlesungen zur Auswahl.
Für mehr Infos musst du die einschlägigen Webseiten und die Prüfungsordnung lesen.

\semester{6}

\modul{I; M; Vertiefung zur Bachelorarbeit}
Weitere Vertiefung nach gleichem Muster wie im fünften Semester in Vorbereitung auf die Bachelorarbeit.

\modul{I; M; Überfachliche Qualifikation}
In dieser Art Nebenfach orientierst du dich fächerübergreifend an Themen deines Interesses, um die fachspezifische Kompetenz zu entwickeln.
Außerdem ist dieses Modul der perfekte Zeitpunkt, eine neue Sprache zu lernen. Japanisch? Arabisch? Russisch? Oder doch ganz einfach Englisch?
Auch hier können Veranstaltungen aus einem Katalog gewählt werden.

\modul{I; M; Bachelorarbeit und Kolloquium}
Als krönenden Abschluss fertigst du die Bachelorarbeit zu einem von dir gewählten Thema an und verteidigst sie in einem Vortrag.
Glückwunsch! Du bist nun offiziell \textit{Bachelor of Science}! Wie wäre es mit einem Master-Studium im Anschluss?

\modul{D; Vertiefung im Nebenfach}
Fortsetzung aus dem 5. Semester.

\modul{D; Basismodul 1, 2 und 3}
Fortsetzung aus dem 5. Semester.

\semester{7. bis 10}

Angehende Diplominformatiker haben nach den sechs Semestern noch vier weitere vor sich.
Im siebten Semester wirst du ein Berufspraktikum absolvieren, im achten und neunten wirst du dann Module auswählen, die dich interessieren, tiefer in die Abgründe des gewählten Themas hinabsteigen und einen \glqq{}Großen Beleg\grqq{} (vergleichbar mit der Bachelorarbeit) schreiben.
Im zehnten Semester wird ausschließlich die Diplomarbeit angefertigt und das war es dann schon!
So schnell kann es gehen.

\begin{figure}[b!]
\centering
\includegraphics[scale=.55]{img/xkcd/eula_faust_20.png}
\caption*{{\small \textit{The only blood these contracts are signed in is from me cutting my hand trying to open the goddamn CD case. (https://xkcd.com/501)}}}
\end{figure}

\changemenucolor{gray}{bg}{named}{ese_fg_color} %background of the menukeys
\changemenucolor{gray}{br}{named}{ese_bg_color} %border of the menukeys
\changemenucolor{gray}{txt}{named}{ese_bg_color} %text of the menukeys
