% Wie erstellt man den Zeitplan?
% 1. man läd sich den HTML Code von ese.ifsr.de herunter
% 2. man fügt folgendes CSS manuell hinzu und bestaunt das Ergebnis:
%  *{font-size:0.99!important}
%	main {margin:0;padding:0;}
%	.row {margin:0;}
%	 header, footer, .sidebar, .panel, section h2, section p, #barrierfree-heading, #timetable-heading  {display:none;}
%	tr, td  {line-height: 1em;}
%	table tbody tr td div.table-spacer {height: 1rem;!important}
%	#content {padding:0;}
%
%	td.data {width: 20%; background: #FDC412}
% 3. man benutzt z.B. wkhtmltopdf um ein PDF daraus zu erstellen:
% wkhtmltopdf -s A4 --no-print-media-type --margin-left 0 --margin-right 0 --margin-bottom 0 --margin-top 0 ESE\ 2016\ _\ iFSR.html zeitplan.pdf
% 4. ????
% 5. Färdsch
%  \vspace*{-6.5em}
\thispagestyle{empty}
  \begin{landscape}

\includegraphics[scale=0.8]{timetable/zeitplan.pdf}%
  \enlargethispage{3em}
%
%  Sofern nicht anders angegeben, finden alle Veranstaltungen im Fakultätsgebäude der Informatik, dem
%  \textbf{Andreas Pfitzmann Bau (APB)}, im Raum \textbf{E023} (Hörsaal direkt am Foyer) statt.
%  Folge im Gebäude einfach den vielen Tutoren in den schönen, roten ESE-2018-Shirts.
%
\begin{center}
  Den aktuellen Zeitplan findest du auch jederzeit unter
  \textbf{https://ese.ifsr.de/}~\link{https://ese.ifsr.de/}
\end{center}
  \end{landscape}
