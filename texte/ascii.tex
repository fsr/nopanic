\addchap{ascii - Das Café in der Fakultät}

\includegraphics[width=\linewidth]{img/ascii.jpg}

Bereits seit 2007 existiert im Gebäude der Fakultät Informatik das ascii, ein Café betrieben von Studenten für Studenten, Mitarbeiter und Besucher, kurzum: für jeden.
Das ascii hat alles was ein richtiges Café so braucht: Kaffee, Kuchen, Bagels so wie alles was Nerds an der Fakultät so brauchen: Koffeinhaltige Kaltgetränke!
Zudem zählt das ascii zu den wenigen Adressen auf dem Campus, in denen man neben Club Mate auch Kolle Mate und Premium Cola erhält.
Hinter dem Tresen stehen Studenten, die gerne an einem Tag in der Woche noch ein paar Stündchen ihrer Freizeit zur Verfügung stellen.

Das ascii wird von einem studentischen Verein betrieben und ist seit seiner Gründung eine zentralen Anlaufstelle für jede und jeden an der Fakultät.
Hier treffen sich Studenten, Mitarbeiter und Professoren um ihre Pausen zu verbringen,
zu arbeiten oder einfach ihren Koffeinhaushalt aufzufüllen.
Auf den gemütlichen Sofas kann man die Zeit wunderbar an sich vorbei streichen lassen,
gemeinsam an Projekten arbeiten, lernen, programmieren oder einfach nur mit seinen Kommilitonen plaudern.
Du kennst noch niemanden an der Fakultät?
Du hast Fragen oder Gesprächsbedarf?
Wenn du ins ascii kommst, wirst du schnell sehen, dass an dem Vorurteil, Nerds seien nicht sozial, absolut nichts dran ist.

Wenn du jetzt Lust bekommen hast, das ascii zu besuchen oder sogar als Mitglied selbst mitzumachen, dann komm doch einfach mal vorbei und sag Hallo!

Weitere Hinweise findest du auf \link{http://www.ascii-dresden.de/}.

\textit{Wir öffnen in der Vorlesungszeit Montag bis Donnerstag von 9 bis 17 Uhr und Freitags von 9 bis 15 Uhr. Dazwischen sind wir aber auch häufig im Café anzutreffen.}
