\addchap{Wichtige Gebäude auf dem Campus}

Auch wenn der Eine oder Andere sich das vielleicht wünscht, wird sich nicht das gesamte Studium in einem Gebäude abspielen.
Stattdessen gibt es einige wichtige Gebäude, die in deinem Studienalltag eine Rolle spielen werden.

\begin{figure}[b!]
    \centering
    \includegraphics[width=\linewidth]{img/slub-lesesaal}
\end{figure}

\minisec{SLUB}
Die Sächsische Landesbibliothek – Staats- und Universitätsbibliothek Dresden, kurz \emph{SLUB} ist das, was an anderen Universitäten einfach Bib heißt.
Mit einer Auswahl von über 12 Millionen Bestandseinheiten bestehend aus Büchern, Magazinen, Filmen, etc.\ ist sie eine der größten (Universitäts-)Bibliotheken Deutschlands.
Neben haptisch erlebbarem Lesewerk bietet die SLUB auch eine umfangreiche Auswahl online verfügbarer Ressourcen, die du dir als Studierender kostenfrei herunterladen kannst.
Doch auch für notorische Nichtleser ist die SLUB aufgrund der vielen ruhigen Arbeitsplätzeein beliebter Aufenthaltsort.
Willst du mit Anderen gemeinsam lernen, gibt es dafür einen weiträumigen Eingangsbereich mit Gruppentischen.
Darüber hinaus können private Gruppenräume reserviert werden. Ein schönes Café, in dem per Mensakarte bezahlt werden kann, rundet das Ganze ab.
Vom APB aus befindet sich die Bibliothek jedoch am anderen Ende des Campus, was aber für Studienanfänger oft kein Problem darstellt. Denn\dots

\minisec{HSZ}
\dots\ gerade die Grundlagenvorlesungen finden nicht im APB, sondern im Hörsaalzentrum (\emph{HSZ}) statt.
In diesem, im Vergleich zum APB doch sehr eintönigen Gebäude, wirst du viele Vorlesungen zu den Grundlagen der Informatik erleben und beim Pendeln zwischen APB und HSZ den einen oder anderen Kilometer ansammeln.
Das HSZ umfasst vier Vorlesungssäle, unter Anderem das Audimax, den größten Hörsaal der Universität und das größte Auditorium Sachsens.
Zusätzlich finden sich hier noch einige Seminarräume, in welchen Übungen stattfinden können. Vor dem \enquote{zentralen Kubus} steht meist ein Mensawagen für den kleinen Hunger zwischendurch.
Auf der anderen Seite des Gebäudes ist die schöne Wiese des HSZ Veranstaltungsort für Feste oder Messen.
Zudem findest du dort den Grillcube, der dich für die Pausen zwischen Vorlesungen mit Burgern versorgt.

\minisec{Mensen}
Wer sein Studium nicht mit Pizzabestellungen bestreiten will, muss das zum Glück auch nicht, denn das Studentenwerk betreibt ein Netz aus 18 Mensen.
Egal in welchem Gebäude der Universität du dich befindest, in der Nähe wird eine Mensa oder zumindest ein Café zu finden sein.
Die Größte der Mensen, die Alte Mensa, befindet sich glücklicherweise direkt zwischen Fakultät und HSZ\@.
Hier findest du eine Vielzahl an täglich wechselnden Hauptgerichten, Salaten und Nachspeisen.
Das Tagesangebot aller Mensen ist auf der Seite des Studierendenwerks~\link{https://www.studentenwerk-dresden.de/mensen/speiseplan/} zu finden.
Für Freunde des späten Frühstücks, oder Studierende mit gestörtem Schlafrhythmus findet sich in der Alten Mensa Mo-Do bis 20 Uhr ein warmes Abendangebot.
Die Auswahl ist jedoch nach 15 Uhr zunehmend eingeschränkter.
Da die Mensa zu Stoßzeiten der Fülle an Menschen kaum standhalten kann, bieten sich Essenszeiten an, die \emph{nicht} direkt nach Ende einer Doppelstunde beginnen.

Sollte euch das Angebot der Alten Mensa nicht zusagt, könnt ihr den Fußweg von 15 Minuten gern in Kauf nehmen. Das Zeltschlösschen ist eine klassische Mensa, die der Alten Mensa in so ziemlich allen Punkten unterlegen ist.

In 15 min zu Fuß könnt ihr ebenfalls die Bio-Mensa U-Boot erreichen. Der Name ist hier Programm und die, in dieser Mensa verwendeten, Zutaten sind aus biologischer und lokaler Erzeugung.
Das in der Mensa U-Boot verwendete Fleisch kommt direkt aus dem Westen Dresdens, was den Preis der Mahlzeiten allerdings nicht unerheblich erhöht.
Wählt ihr die fleischlose Alternative, sollten sich die Preise nicht sonderlich von denen anderer Mensen unterscheiden.

Auch am Wochenende kannst du dem Kochen entkommen, denn die Mensa Siedepunkt hat auch an Samstagen und Sonntagen geöffnet.
Gerade nach einer produktiven Lerneinheit in der SLUB ist diese ganz bequem mit einem einfachen Wechsel der Straßenseite zu erreichen.

FSR Geheimtipp:
Ohne Zweifel die beliebteste Mensa auf dem Campus, schenkt man den Aussagen Dresdner FSRlingen glauben, ist Firat.
Es handelt sich hierbei um einen unabhängigen Dönerladen, doch durch den effizienten Umgang mit überdurchschnittlich hohem studentischen Andrang wird er gern auch scherzhaft als Mensa bezeichnet.
Das Dönerhaus liegt knapp 10 Minuten Fußweg vom APB entfernt und lockt mit guter Qualität, gemündlichem Ambiente und einer vergleichsweise großen Auswahl an Gerichten.
e nach einer produktiven Lerneinheit in der SLUB ist diese ganz bequem mit einem einfachen Wechsel der Straßenseite zu erreichen.

\minisec{Seminargebäude}
Im Seminargebäude, an der besonders schönen Fassade erkennbar, finden die Sprachkurse des LSK statt.
Für alle, die über einen einfachen Sprachkurs hinausgehen wollen, werden zusätzlich Seminare zur Kultur und Politik ausgewählter Länder und Regionen angeboten.
Weitere Informationen findest du auf der Website des LSK~\link{https://tu-dresden.de/gsw/slk/lsk}.
Das Gebäude steht direkt neben der SLUB, ca.\ 20 Minuten vom APB entfernt.
