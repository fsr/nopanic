\documentclass[12pt]{article}
%\documentclass[12pt, twocolumn]{article}
%\usepackage[a5paper]{geometry}   %zweispaltig und a5, aber format folgt später
\usepackage[ngerman]{babel}
\usepackage[utf8]{inputenc}
\usepackage[parfill]{parskip}
\usepackage{hyperref}
\usepackage{newclude}
%\usepackage[margin=0.5in]{geometry}

\begin{document}

% ----------
% Coverseite
% ----------
NO PANIC \\
2014
\newpage
% ----------


\tableofcontents


% vielleicht auch nicht direkt als section...
\section{Danke an ...}

%TODO: Tutorennamen hier einfügen
Tutor1, Tutor2, Tutor3, ...

\include*{Vorwort}
\include*{Grusswort}
\include*{Studienbetrieb}
\include*{Checkliste}
\include*{Moduluebersicht}
\include*{F1Help}
\include*{Dozenten}
\include*{auditorium}
\include*{FFFI}
\include*{FSR}
\include*{CD}
\include*{ASCII}
\include*{ZIH}
\include*{Glossar}
\include*{Lesezeichen}


\section{Campusplan}
%TODO: Lageplan hier einfügen

%TODO: Werbung (von Citrix?) hier einfügen


% ----------
% Backcover
% ----------
\newpage
\textbf{Herausgeber} \\
Fachschaftsrat Informatik der TU Dresden
Nöthnitzer Straße 46, 01062 Dresden

\textbf{Redaktion} \\
% noch einfügen

Mitarbeit an der nächsten Version unter \url{github.com/fsr/No-Panic} ist immer willkommen!
\end{document}


% Allgemeine TODOs
% ----

