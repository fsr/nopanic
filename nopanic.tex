\documentclass[12pt]{book}
%\documentclass[12pt, twocolumn]{article}
\usepackage[a5paper]{geometry}   %zweispaltig und a5, aber format folgt später
\usepackage[ngerman]{babel}
\usepackage[utf8]{inputenc}
\usepackage[parfill]{parskip}
\usepackage{hyperref}
\usepackage{newclude}
\usepackage{fancyhdr}
\usepackage{xcolor}
%\usepackage[margin=0.5in]{geometry}

%"Kapitel" ist etwas doof, aber so ist da ein newline :/
\addto{\captionsngerman}{\def\chaptername{}} 

%Seitenzahlen unten sind cooler, außerdem sind die Header nutzlos
\pagestyle{fancy}
\lhead{}
\rhead{}
\cfoot{\thepage}

%Headerline in *book* entfernen
\renewcommand{\headrulewidth}{0pt}
\renewcommand{\footrulewidth}{0pt}

%Um was geht's hier eigentlich?
\title{NO PANIC}
\date{2014}

\begin{document}

% ----------
% Coverseite
% ----------
\pagecolor{green}
\color{white}
\maketitle
\color{black}
\pagecolor{white}
% ----------

\tableofcontents

%\newpage
\vspace{1cm} %provisorisch
\textbf{Danke an ...}

%TODO: Tutorennamen hier einfügen
Tutor1, Tutor2, Tutor3, ...

\include*{Vorwort}
\include*{Grusswort}
\include*{Studienbetrieb}
\include*{Checkliste}
\include*{Moduluebersicht}
\include*{F1Help}
\include*{Dozenten}
\include*{auditorium}
\include*{FFFI}
\include*{FSR}
\include*{CD}
\include*{ASCII}
\include*{ZIH}
\include*{Glossar}
\include*{Lesezeichen}


\chapter{Campusplan}
%TODO: Lageplan hier einfügen

%TODO: Werbung (von Citrix?) hier einfügen


% ----------
% Backcover
% ----------
\newpage
\pagecolor{green}
\textbf{Herausgeber} \\
Fachschaftsrat Informatik der TU Dresden
Nöthnitzer Straße 46, 01062 Dresden

\textbf{Redaktion} \\
% alphabetisch sortiert
Julius,
Kilian,
Manuel,
Niklas,
Sebastian

Mitarbeit an der nächsten Version unter \url{github.com/fsr/nopanic} ist immer willkommen!
\end{document}


% Allgemeine TODOs
% ----

