\chapter{Press F1 For Help}

\textit{Seminar Groups / Fachschaftsrat / Studierendenrat / Student Advisory Service / Studentenwerk 
/ Dean of Studies / Studying with Disabilities and Chronic Illness / Examination Office}


Should difficulties or questions arise at any time during your studies, don't be afraid to ask for help in time.
The principle applies: It is better to ask once too often than once too little, because, as we all know, asking costs nothing.
And let's be honest, many people before and after you will have faced and still face similar problems.
Fortunately, there are numerous ways to clarify questions and get help.
The respective lecturers and tutors are always happy to answer technical questions.
Do not hesitate for long and raise your arm, you will see that many of your fellow students will be grateful to you.
Because even if everyone around you is nodding thoughtfully while a complicated differential equation is being solved at the front of the blackboard -- the truth is that most of them, just like you, have no idea.
Are you still a bit shy at the beginning? Then remember your question and go to the lecturer before or after the lecture.

With time, you will certainly get to know fellow students from older semesters whom you can ask for help.
This applies not only to questions about the content of a lecture, but also to the course of study.
How does the crediting of this module work? When can I register for exams? Where do I have to report if I was sick during an exam?
If in doubt, contact the student council!
There you will get a confidential and usually quick answer to your question or at least the contact who can give it to you.

\section*{What to do with my questions?}
\begin{enumerate}
\item Ask your fellow students.
\item Get advice from your seminar group mentor.
\item If neither of these help, please contact the FSR. You can do this in person or by e-mail.
\end{enumerate}


\label{sec:seminargruppen}
\section*{Seminar groups}
To help you find your way around quickly at the beginning of your studies, you will be assigned to a seminar group together with other students for the first semester. You will share a common timetable. In the tutorials you will see familiar faces again and again, with whom you can form study groups. Often new friendships are formed. Your seminar group mentor is always available as a direct contact person for questions or problems. In the course of the first semester, there are several meetings where important information about the course and organization of the studies is conveyed. Therefore, you should not miss these meetings.

\label{sec:fachschaftsrat}
\section*{Student Representative Council (FSR)}

\includegraphics{img/fsr_logo.svg}

The Student Representative Council is your student representative at the faculty level.
It is elected annually and currently consists of 18 members of the Fachschaft Informatik, which you are a part of. 
The FSR office is located on the first floor in room APB/E017.
To get in touch, you can also just write an email to 
\href{mailto:fsr@ifsr.de}{fsr@ifsr.de}.

\subsection*{What can we do for you?} 

The FSR is the first point of contact for you if you have problems or questions about your studies.
It offers many useful tips and assistance, such as exam collections \url{https://ftp.ifsr.de/klausuren},
protocols of oral exams \url{https://ftp.ifsr.de/komplexpruef/} 
and consulting services, which you are welcome to make use of at any time.

In addition to supporting you in your studies, the FSR also tries to make your life beyond the university a bit more pleasant by offering social and cultural activities.
For example, the FSR usually organizes game nights with board, card and digital games once a month in the faculty.
It co-organizes Christmas parties, barbecues, hikes, sports tournaments and much more.
Last but not least, it plans the ESE together with many helpers and has created this booklet for you.

The FSR is also a central component of student participation in university committees, boards and commissions. There, for example, it has a direct influence on the revision of study regulations and the appointment of professors at our faculty. It helps to control and improve the quality of teaching, for example by evaluating lectures.

The FSR office offers the possibility to print a small number of pages. In addition, it offers various equipment and materials for rental
\url{https://www.ifsr.de/service/geraete-ausleihen}.
Among other things, you can borrow various Raspberry Pi, an Oculus Rift and Lego Mindstorms robots. 
Just drop by and bring your ID with you the first time so that you can be issued a loan certificate right away.

\subsection*{How can you get involved?} 

In many areas, the university is managed democratically from the bottom up.
This democracy thrives on participation and on people who implement ideas, projects and visions.
This has worked successfully for many decades.
To keep it that way, the FSR, just like many other bodies and groups at the university, needs active, committed, motivated members and supporters at all times.

\subsection*{Student representation is what you make of it}

\includegraphics{img/f1_neu.png}

Even if you don't want to stand for election yourself, you can do something for your student council.
The first step is to vote. This costs very little time and is important to legitimize the FSR and to give it support for the representation of student interests.
And how about, for example, participating in the ESE yourself next year?
Or maybe teach a programming course?
We are always looking for organizational talent and contributors for individual events like the sports tournaments or the Lange Nacht der Wissenschaften.

So if you don't want to be officially elected for the whole year, you can always help and contribute.
There are enough tasks or ideas and you are welcome to bring your own.

Normally, the FSR meets every Monday at 18:45 in the large council room (APB/1004) to discuss various topics, such as committee work, upcoming events and demonstrations, but also problems and developments at the faculty and university. You are cordially invited and can just drop by, because the meetings are usually open to the public. On the website of the FSR \url{https://www.ifsr.de}  you can find the minutes of the meetings and a lot of other useful information. Due to the current situation regarding Covid-19, meetings are also held online via Big Blue Button  \url{https://www.ifsr.de/events/fsr-sitzung}.

\section{Student Council (StuRa)}\label{sec:stura}

The Student Council, often known as the Allgemeiner Studierendenausschluss (AStA) and the Studierendenparlament (StuPa) in other German federal states, is the body that is superior to the student councils and represents the interests of all students at the TU Dresden.

However, it also offers a range of services, including: 


\begin{itemize}
\item BAföG and social counseling
\item legal advice
\item Counseling for foreign students
\item Advice for students with children
\item Advice on applications and funding opportunities
\item Sale of tickets for various cultural events
\end{itemize}

Detailed information can be found on the StuRa website \url{https://www.stura.tu-dresden.de}.

% Wird der gedruckte Spirex eigentlich noch aktualisiert? Die PDF ist von 2014/15 :/

\section{Student counseling}

Sometimes studying does not go smoothly. Individual students have orientation difficulties or problems coping with the requirements, especially at the beginning of their studies.
The Student Advisory Service supports you with information in all phases of your studies. The counseling covers, for example, questions about exams and exam preparation, specialization options, changing majors, or questions about scheduling.
The TU Dresden offers a general, inter-faculty student advisory service \url{https://tu-dresden.de/studium/im-studium/beratung-und-service/zentrale-studienberatung} 
which can help you, for example, with doubts about the choice of study program, exam anxiety or similar concerns.
For all questions and problems specific to a study program, our faculty has its own counseling services with contacts among students and non-students
\url{https://tu-dresden.de/ing/informatik/studium/beratung}.

\minisec{Studierendenwerk (StuWe)}
\label{sec:stuwe}
The Studierendenwerk not only runs the dormitories and provides good and inexpensive meals in the dining halls and cafeterias, but also offers extensive support and assistance for students. These include:
\begin{itemize}
\item Legal and social counseling
\item Psychosocial counseling
\item Processing of BAföG applications
\item Assistance for students with disabilities
\item Pregnancy and childcare during studies
\end{itemize}

Further information about the tasks and offers of the StuWe can be found on the internet 
\url{https://www.studentenwerk-dresden.de/}.


\minisec{Dean of Studies and Student Representative}
There are many different offices in the Faculty. The highest and most important is the role of the dean of the faculty and his deputy, the vice dean. However, for teaching and thus your studies, the so-called Dean of Studies is of great importance. He is responsible for the affairs of teaching in the faculty, mediates between students and teachers and helps with problems with the study in general.

\pagebreak
\textbf{Dean of Studies for German-language degree programs}\\
Prof. Dr. rer. nat. habil. Gerhard Weber \\
Office: APB/1055 \\
Phone: (0351) 463-38477 \\
E-mail: gerhard.weber@tu-dresden.de

\textbf{Dean of Studies for English-language degree programs}\\
Prof. Dr. Christof Fetzer \\
Office: APB/3104 \\
Phone: (0351) 463-39709 \\
E-Mail: christof.fetzer@tu-dresden.de

\textbf{Representative for the teaching degree programs}\\
Prof. Dr. rer. nat. Nadine Bergner \\
Office: APB/2096 \\
Phone: (0351) 463-38306 \\
E-Mail: nadine.bergner@tu-dresden.de

\minisec{Study with Disability and Chronic Illness}

At the TU Dresden we are always working on a barrier-free design of the studies as well as the study environment. If you have a health impairment, special questions and issues around your studies often arise. Numerous support and counseling services are available to help you participate in your studies in a way that is appropriate for you~\link{https://tu-dresden.de/studium/rund-ums-studium/studieren-mit-beeintraechtigung}.


\refstepcounter{dummy}\label{sec:pruefungsamt}
\minisec{Prüfungsamt (PA)}
The Prüfungsamt is responsible for registering students for examinations, announcing results, managing examination files and other matters relating to examinations. You will find a lot of information, forms and frequently asked questions on the pages of the Prüfungsamt~\link{https://tu-dresden.de/ing/informatik/studium/pruefungsorganisation}. Examination schedules and registration deadlines are also announced there. During the open office hours, you can also submit your applications and questions without an appointment in room APB/3039.

\textbf{Contact}

Phone: (0351) 463-38230

E-Mail: \href{mailto:pruefungsamt.inf@tu-dresden.de}{pruefungsamt.inf@tu-dresden.de}

