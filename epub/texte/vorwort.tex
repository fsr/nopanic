\chapter{What is this?}

Hello! You've already made it this far. Wow!

As you might find out in the course of your now beginning studies, 
one of the hardest tasks is to get a computer scientist to read the manual or the documentation. 
Maybe it's a quirk we have, or maybe we just prefer to tinker with a problem for days on end rather than spend half an hour looking it up in the manual. 
That's why you'll often hear the phrase: \enquote{RTFM} -- Read the fucking manual -- 
in forums or on \href{https://stackoverflow.com/}{StackOverflow}.

But the good news is, if you are reading this, you are already counteracting this prejudice. 
Because this booklet is a small manual for your studies. It tells you how the university works, how courses are structured, how to register for exams, what the campus has to offer and much more. 

% Like any good handbook, it contains far too much information that you can hardly use or process at once.

% That's why it's enough to read only the chapters marked with \keys{must read} 
% and look up the rest later or use the FAQ section on page \pageref{sec:faq} to find the right answers. 

This book is designed to help you with information and answers throughout most of your studies. 
So keep it and take it out every now and then. Like every good handbook :)

And now away with the book! The Erstsemestereinführung (ESE) is waiting for you, which will explain the most important things to you in the coming week. Enjoy this time and get ready for the even more exciting one that awaits you now!

\textbf{All that remains is to say: Enjoy the ESE and your studies!}

% Martin, 31. August 2022
% In dem Handbuch sollten einfach nur die wichtigsten Dinge genau einmal stehen (DRY), 
% damit man es einfach einmal lesen kann.
