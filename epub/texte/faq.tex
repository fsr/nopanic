\chapter{Frequently Asked Questions} \label{sec:faq}

\textbf{What do I have to do during the semester?}

Basically nothing except re-registering at the end to continue enrollment.

\textbf{What should I do during the semester?}

You should attend the tutorials and lectures in order to pass the exams at the end.

\textbf{How do I enroll in exams?}

For the written exams in the compulsory modules, the registration is done via \href{https://jexam.inf.tu-dresden.de/}{jExam}.
At the end of the lecture period, there is a list with all exams you can enroll in on there.

\textbf{How can I withdraw from exams?}

You can withdraw from written exams via jExam up to three working days without giving a reason. If you are ill, you can still withdraw from the exam afterwards. The sickness certificate must then be submitted to the examination office as soon as possible.

\textbf{I have failed an exam, when do I have to retake it?}

Module exams that have not been passed must be retaken once within one year. If you fail, you must take a second retake at the next possible examination date. After that, the module exam is finally considered as failed. COVID-19 may have special regulations in this regard. It is important to keep yourself informed, e.g. 
\href{https://tu-dresden.de/tu-dresden/gesundheitsmanagement/information-regarding-covid-19-coronavirus-sars-cov-2}{on the Corona information page of the TU Dresden}. 
By the way: A module examination consisting of several examinations can be passed if the grades are weighted accordingly, even if one of the examinations is not passed.

\textbf{How long may one study at all?}

In short: standard period of study + 4 semesters. However, there are some possibilities, such as vacation and committee semesters, to extend this duration. Keep in mind, however, that long-term tuition fees are due after the 14th semester.

\textbf{How do I build my schedule?}

For the beginning you will receive ready-made schedules from us, from which you can then simply choose a schedule on the day of enrollment. From the second semester on, the study schedules offer a recommendation which modules you should attend in which semester. You pick the courses from the catalog that you want to attend. Then you pick dates and try to avoid clashes.

\textbf{What is AQua?}

This abbreviation stands for Allgemeine Qualifikation.
It is a part of your \hyperlink{lec:aqua}{studies}.

%\minisec{Was sollen diese ganzen Portale?}
%In guter Tradition gibt es für jeden Zweck mindestens 5 verschiedene Portale. Das muss so. Für dich sind eigentlich nur folgende wichtig:
%\begin{itemize}
%\item jExam für die Einschreibung in Lehrveranstaltungen und Prüfungen
%\item Selma~\link{https://selma.tu-dresden.de} für das Herunterladen der Immatrikulationsbescheinigung
%\item OPAL wird besonders in der Online-Lehre häufig für die Einschreibung und Organisation von Lehrveranstaltungen genutzt.
%\end{itemize}

\textbf{How does the wifi work?}

The ZIH \href{https://tu-dresden.de/zih/dienste/service-katalog/arbeitsumgebung/zugang_datennetz/wlan-eduroam}{has instructions} for many systems on how to set up eduroam.
If the instructions do not solve your problem, the ServiceDesk will help you.

\textbf{What do I have to do to get BAföG?}

The \hyperlink{sec:stuwe}{Studierendenwerk} is responsible for processing the applications and also offers corresponding advice.
You can also contact the \hyperlink{sec:stura}{StuRa}.

\textbf{How and when do I register for university sports?}

This works through the Universitätssportzentrum, which manages \hyperlink{sec:sport}{all courses}.
% Auf der Seite des Universitätssportzentrums gibt es eine Liste mit allen Angeboten und Terminen, an denen die Einschreibung beginnt. Dann kannst du auf der Seite das Formular schnell ausfüllen und bist eingeschrieben.

\textbf{How and when do I register for language courses?}

For this, there is the website of the Lehrzentrum Sprachen und Kulturen -- \hyperlink{sec:sprache}{LSK in short}.

\textbf{How do I make friends?}

During the ESE you have the chance to get in touch with many people who are probably facing a similar problem as you. Apart from that from time to time there are \hyperlink{cha:veranstaltungen}{events} where you can get to know fellow students. 
Or you can look around outside of your studies. After all, your life doesn't just consist of studying ;)

\textbf{Who or what is the FSR?}

The \hyperlink{sec:fachschaftsrat}{Fachschaftsrat, or FSR in short,} 
represents you and your interests at the university. 
In addition, we also organize events such as the the ESE. 

\textbf{What can I do besides my studies?}

You can volunteer in \hyperlink{sec:hsg}{university groups}, the \href{https://www.agdsn.de}{AG DSN}, 
in the FSR, or anywhere else.
Alternatively, you can of course earn money and fill one of the many positions.
Many job offers are distributed to students via a 
\href{https://lists.ifsr.de/mm3/postorius/lists/extern.ifsr.de/}{mailing list}.
If there is nothing there for you, check out the 
\href{https://www.stav-dresden.de}{STAV (studentische Arbeitsvermittlung)}. 

\textbf{Is the semester ticket the ultimate power tool?}

Unfortunately, no. 
There are a handful of restrictions (e.g. you are only allowed to take bicycles for free at certain times),
but they are explained to you on \href{https://www.stura.tu-dresden.de/semesterticket}{these StuRa pages}.

\textbf{What do I have to do with my new apartment?}

You should probably pay your rent regularly. 
Otherwise, don't forget to register yourself in any of the \hyperlink{sec:ummelden}{Bürgerbüros in Dresden}!
% Du musst einen Wohnsitz anmelden. Achtung: Die Stadt Dresden erhebt eine Zweitwohnsitzsteuer, es ist also sinnvoll, den Hauptwohnsitz nach Dresden zu verlegen. Zugezogene können außerdem von einer Umzugsbeihilfe profitieren.

\textbf{Where do I learn programming?}

In the lectures, unfortunately, not at all. 
There you usually only get a \hyperlink{sec:aud}{short crash course}, like for example in algorithms and data structures for the language C.
However, the FSR offers \href{https://kurse.ifsr.de}{programming courses for almost all languages} that you will need during your studies. 
You can also take programming courses through LinkedIn Learning,
that are 
\href{https://www.slub-dresden.de/forschen/datenbanken-zeitschriften-normen/datenbanken/trainingsvideos}{provided by the SLUB}.

\textbf{When do I learn game development?}

In the degree program, probably not to at all. 
There are no real events for this apart from a few complex internships later in the course of studies. 
You learn the basics of graphics rendering in \hyperlink{lec:ecg}{Einführung in die Computergrafik}, but that's it. :(

\textbf{What is recursion?}

Have a look at the question \emph{„What is recursion?“}

\textbf{What kind of books do I need?}

Actually none, but if you want to read something in more detail or if you didn't quite understand something, there are enough books and eBooks on all topics in \hyperlink{sec:slub}{the SLUB}.

%\minisec{Wo kann ich Dinge drucken?}
%Es gibt viele Orte an denen gedruckt werden kann. Zum einen bietet bspw. die SLUB einen Druckservice an. Zum anderen gibt es aber auch auf dem Campus und in der ganzen Stadt Copyshops.

\textbf{How do I get old exams?}

There is a collection on the FTP server of the FSR~\link{https://ftp.ifsr.de/klausuren} (only accessible from the university network or VPN), but some chairs also put some on their websites.

\textbf{What is this s-number?}

The s-number is your ZIH login. This system has been changed in the meantime and your ZIH login now consists of letters and numbers, as described on the ZIH website~\link{https://tu-dresden.de/zih/dienste/service-katalog/zugangsvoraussetzung}. Previously, the login consisted only of a 7-digit number preceded by an \enquote{s} -- hence \enquote{s-number}.

\textbf{What are those yellow bicycles standing around for?}

The yellow bikes belong to the bike rental system MOBIbike, which is an offer of the Dresdner Verkehrsbetriebe AG (DVB) and nextbike. You can rent the MOBIbikes and even nextbikes in most other German nextbike cities using a discount which is included in your semester ticket. How exactly this works and more details can be found on the web pages of the StuRa~\link{https://www.stura.tu-dresden.de/nextbike}.

\textbf{Why is there no water in the pond behind the APB?}

A new building for the Deutsche Zentrum für Luft- und Raumfahrt
(DLR) is to be built on the vacant area behind the Andreas-Pfitzmann-Bau.
For this the pond had to be \hyperlink{sec:apb}{drained}.
